\documentclass{beamer}
%\usetheme{Copenhagen}
%
% Choose how your presentation looks.
%
% For more themes, color themes and font themes, see:
% http://deic.uab.es/~iblanes/beamer_gallery/index_by_theme.html
%
\mode<presentation>
{
  \usetheme{Madrid}      % or try Darmstadt, Madrid, Warsaw, ...
  \usecolortheme{default} % or try albatross, beaver, crane, ...
  \usefonttheme{default}  % or try serif, structurebold, ...
  \setbeamertemplate{navigation symbols}{}
  \setbeamertemplate{caption}[numbered]
} 
\usepackage{amsmath}
\usepackage[english]{babel}
\usepackage[utf8x]{inputenc}
\usepackage[round]{natbib}
\usepackage{bibentry}
%\usepackage{biblatex}
%\bibliography{references}
\nobibliography*
\setbeamercolor{bibliography entry author}{fg=black}
\setbeamercolor{bibliography entry title}{fg=black} 
\setbeamercolor{bibliography entry location}{fg=black} 
\setbeamercolor{bibliography entry note}{fg=black} 
\DeclareMathOperator*{\argmax}{arg\,max}

\def\one{\mbox{1\hspace{-4.25pt}\fontsize{12}{14.4}\selectfont\textrm{1}}} % 11pt   



\title{Budget-constrained Recommendation of a Set of Alternatives for Common Use}
\author{Silvan Hungersb\"{u}hler \& Haukur J\'{o}nsson \& Grzegorsz Lisowski \& Max Rapp}
\date{17/05/2017}

\begin{document}
\begin{frame}
	\titlepage
\end{frame}

\begin{frame}{Outline}
We seek to find good ways for a group of agents to spend a limited budget on a set of alternatives which they all have to consume and over which they have different preferences. 
	\begin{itemize}
		\item Motivation: Two Scenarios
		\item The Knapsack Problem for Multiple Agents
		\item Formal Framework
		\item Desirable Properties of Recommendation Sets
		\item Budgeted Voting Rules
		\item Results: Rules \& Property Satisfaction
		\item Future Work
	\end{itemize}


\end{frame}

\begin{frame}{Motivation: A problem with the News Media}

News Media are vulnerable to the following vicious cycle: People are biased. They are more likely to consumn media content that confirms their biases. Thus media have an incentive to feed their consumers' bias. If they follow it, this can entrench division and erode the commonly accepted factual base. If they don't, a part of the population may come to believe that their world view is not represented by the media and become increasingly distrusting of the news.



\end{frame}

\begin{frame}{A Common Core}

Can we devise a transparent,  intersubjectively justified method to select a common core of issues to which everybody should pay attention?
  
	
\end{frame}
\begin{frame}{Scenario I: An Essential Readings News Recommendation Algorithm}

Online social networks often recommend news articles to their users based on an algorithm that infers preferences from user's past behavior and demographic properties. Such a News Feed may have little overlap for people belonging to different social or political spheres.


Imagine a news aggregator that asks users for an amount of time they want to read everyday, scrapes their social network accounts and based on this information creates a list of essential readings of the desired length.




  

\end{frame}

\begin{frame}{Scenario II: A  Newspaper Page}

A newspaper with a diverse readership has recently lost readers on one side of the political spectrum who feel that the medium does not report on the issues they care about. Responding to this criticism they would like to create a page that reflects preference data they collect from their readers.

How can they fill the page in a way that is a good compromise given their voters diverse preferences over issues?
		
\end{frame}

\begin{frame}{The Knapsack Problem}
At a higher level of abstraction, we seek to find good ways for a group of agents to spend a limited budget on a set of alternatives which they all have to consume and over which they have different preferences. 	

For the single agent case, this problem is the well-known Knapsack Problem: How can I find the combination of items that fits in my knapsack which is most valuable for my camping trip?

However, if the knapsack is packed for a group with different preferences, there is no simple way to maximize utility. Thus a voting rule is needed.

\end{frame}

\begin{frame}{Formal Framework}
We have a set of \emph {news items} $A=\{a_1,...,a_m\}$, 
a set of \emph {recommended items} $W\subseteq A$, 
a set of \emph {consumers} $N=\{n_1,...,n_n\}$, 
a \emph {profile of preferences} over the set of items $\mathcal{R}\in \mathcal{L}^n$ 
and a \emph {budget} $B\in \mathbb{R}_{\geq 0}$. %should be \mathbb

We derive utilities from $\mathcal{R}$ employing a utility function $u_i:\mathcal{L}^n \times N \times  A \rightarrow \mathbb{N}$ assigning each news item its Borda score for the preference order of reader i. This function is extended to $u(a)=\sum_{i=1}^n a$, $u_i(W)=\sum_{a\in W}u_i(a)$ and $u(W)=\sum_{a\in W}u(a)$ for $W\in \mathcal{P}(A)$. Likewise each news item has a specific \emph{cost} $C: A \rightarrow \mathbb {N}$ that extends to sets by $C(W)=\sum_{a\in W}C(a)$.

\end{frame}

\begin{frame}{Desirable Properties: Axioms}

Here should be a table with all the axioms for which we have proofs



\begin{block}{Reference}
	%\bibentry{Elkind2017}
\end{block}

\end{frame}

\begin{frame}{Desirable Properties: Utility Maximization}

Let $\mathcal {W_B}$ be the set of all elements of $\mathcal{P}(A)$ s.t. $C(W)\leq B$. Utility is maximized if:

\[
W=\argmax_{W'\in \mathcal{W}_B}(u(W')) 
\]

	
\end{frame}

\begin{frame}{Desirable Properties: $\theta$-Minority Consistency}
	
A recommendation set is $\theta$-minority consistent if \[a\in W\text{ whenever }\frac {\mathcal{N}_{a\succ b}}{N}\geq \theta \text{ for all } b\in A\setminus \{b\} \]
	
\end{frame}

\begin{frame}{Desirable Properties: $\delta$-Equality}
	
	Define the Gini-coefficient of a recommendation set as follows: 
	
	\[G(W)=\frac{\displaystyle{\sum_{i=1}^n \sum_{j=1}^n \left| u_i(W) - u_j(W) \right|}}{\displaystyle{2n* u(W)}}\]
	
	For $\delta\in [0,1]$, a recommendation set is $\delta$-egalitarian if $G(W)\leq \delta$. 
	
\end{frame}

\begin{frame}{Voting Rules: Knapsack}

This rule takes the voters preference orders and maximizes utility given the budget:
\[
\max_{W\in\mathcal{ W_B}} u(W)
	\]
	
\end{frame}

\begin{frame}{Voting Rules: Commitee of $\theta$-Winners}
	
We define a news item's $\theta$-score as follows: \[\theta(a)=\min_{b\in A} \frac{\mathcal{N}_{a\succ b}}{N}\]

The commitee of $\theta$-winners is then chosen as follows:

\[W_1=\{\argmax_{a\in A}\theta(a)\}\cap \mathcal{W}_B\]
\[W_n=W_{n-1}\cup \{\argmax_{a\in A\setminus W_{n-1}}\theta (a)\}\cap \mathcal{W}_{B-C(W_{n-1})}\]
\[W=W_{|A|}\]

\end{frame}

\begin{frame}{Voting Rules: Further Rules}
	Further we tested budgetized versions of the k-Plurality, k-Borda and k-Copeland rules. 
\end{frame}

\begin{frame}{Methods: Proofs \& Data-Generation}

We proved satisfaction or failure of satisfaction of the axioms. In addition we simulated the behaviour of the voting rules with respect to the desirable properties and the failed axioms. 

For this purpose, since unfortunately good real world data was hard to find, we generated datasets randomly to our needs. We generated one dataset of small profiles with 5000 readers and 10 news items and one dataset of large profiles with 5000 readers and 100 news items. In addition, for each dataset we used various procedures to generate fully random profiles as well as profiles of several degrees of "fragmentization" from 1 to 100 clusters of preference orders.

\end{frame}

\begin{frame}{Methods: Statistical Analysis}
	Greg should write something here.
\end{frame}

\begin{frame}{Results I: Axioms}
	
	Here a table with the rules and which axioms they satisfy
	
\end{frame}

\begin{frame}{Results II: Utility Maximization}
	
\end{frame}

\begin{frame}{Results III: $theta$-Minority Consistency}
	
\end{frame}

\begin{frame}{Results IV: $\delta$-Equality}
	
\end{frame}

\begin{frame}{Results V: An Overall Best Rule?}
	
\end{frame}

\begin{frame}{Future Work}

\begin{itemize}
	\item The cost and utility functions could be used to express dependencies (some items form series of articles, cost might vary depending on reader)
	\item Implement slack: in many application leftover budget also has a utility
	\item Find, collect or mine real data (webscraping...).
	\item Build a ready-to-use implementation!
	
\end{itemize}
	
\end{frame}






%\bibliographystyle{apacite}
%\bibliography{references}
\end{document}