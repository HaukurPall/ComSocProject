\documentclass{beamer}
%\usetheme{Copenhagen}
%
% Choose how your presentation looks.
%
% For more themes, color themes and font themes, see:
% http://deic.uab.es/~iblanes/beamer_gallery/index_by_theme.html
%
\mode<presentation>
{
  \usetheme{Madrid}      % or try Darmstadt, Madrid, Warsaw, ...
  \usecolortheme{default} % or try albatross, beaver, crane, ...
  \usefonttheme{default}  % or try serif, structurebold, ...
  \setbeamertemplate{navigation symbols}{}
  \setbeamertemplate{caption}[numbered]
} 

\usepackage[english]{babel}
\usepackage[utf8x]{inputenc}
\usepackage{apacite}
\usepackage[round]{natbib}



\title{Budget-constrained Recommendation of a Set of Alternatives for Common Use}
\author{Silvan Hungersb\"{u}hler \& Haukur J\'{o}nsson \& Grzegorsz Lisowski \& Max Rapp}
\date{17/05/2017}

\begin{document}
\begin{frame}
	\titlepage
\end{frame}

\begin{frame}{Outline}
We seek to find good ways for a group of agents to spend a limited budget on a set of alternatives which they all have to consume and over which they have different preferences. 
	\begin{itemize}
		\item Motivation: Two Scenarios
		\item The Knapsack Problem for Multiple Agents
		\item Desirable Properties of Recommendation Sets
		\item Budgeted Voting Rules
		\item Results: Rules \& Property Satisfaction
		\item Future Work
	\end{itemize}


\end{frame}

\begin{frame}{Motivation: A problem with the News Media}

News Media are vulnerable to the following vicious cycle: People are biased. They are more likely to consumn media content that confirms their biases. Thus media have an incentive to feed their consumers' bias. If they follow it, this can entrench division and erode the commonly accepted factual base. If they don't, a part of the population may come to believe that their world view is not represented by the media and become increasingly distrusting of the news.



\end{frame}

\begin{frame}{A Common Core}

Can we devise a transparent,  intersubjectively justified method to select a common core of issues to which everybody should pay attention?
  
	
\end{frame}
\begin{frame}{Scenario I: An Essential Readings News Recommendation Algorithm}

Online social networks often recommend news articles to their users based on an algorithm that infers preferences from user's past behavior and demographic properties. Such a News Feed may have little overlap for people belonging to different social or political spheres.


Imagine a news aggregator that asks users for an amount of time they want to read everyday, scrapes their social network accounts and based on this information creates a list of essential readings of the desired length.




  

\end{frame}

\begin{frame}{Scenario II: A  Newspaper Page}

A newspaper with a diverse readership has recently lost readers on one side of the political spectrum who feel that the medium does not report on the issues they care about. Responding to this criticism they would like to create a page that reflects preference data they collect from their readers.

How can they fill the page in a way that is a good compromise given their voters diverse preferences over issues?
		
\end{frame}

\begin{frame}{The Knapsack Problem}
At a higher level of abstraction, we seek to find good ways for a group of agents to spend a limited budget on a set of alternatives which they all have to consume and over which they have different preferences. 	

For the single agent case, this problem is the well-known Knapsack Problem: How can I find the combination of items that fits in my knapsack which is most valuable for my camping trip?

However, if the knapsack is packed for a group with different preferences, there is no simple way to maximize utility. Thus a voting rule is needed.

\begin{frame}{Formal Framework}
	
\end{frame}

\end{frame}
\end{document}