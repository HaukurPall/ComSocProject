\documentclass[10pt,a4paper]{article}
\usepackage[utf8]{inputenc}

\title{
  Framework Proposal\\ \large COMSOC Project
\\
  \large Silvan Hungerbuehler}

\usepackage{mathptmx} % "times new roman"
\usepackage{amssymb}
\usepackage{amsmath, amsthm}
\usepackage{amsfonts}
\usepackage{enumitem}
\usepackage{verbatim}
\usepackage{hyperref}
\usepackage{comment}
%\usepackage[margin=1in]{geometry}
\usepackage{float}
\def\one{\mbox{1\hspace{-4.25pt}\fontsize{12}{14.4}\selectfont\textrm{1}}} % 11pt    
\usepackage{bm}

\newcommand{\haukur}[1]{\textcolor[rgb]{.8,.33,.0}{[TB: #1]}}% prints in orange

\usepackage[normalem]{ulem}
\date{}
\begin{document}
\maketitle
\paragraph{Basic Idea}
We construe the problem of giving a media recommendation to a group of people as the problem of finding that recommendation that maximizes readers' value while satisfying some budget side-constraint. Maximizing value (which could probably be recast as minimizing misrepresentation) will require coming up with some metric based on the profile and the recommendation; my preliminary suggestion for this is to use the Borda score.
\paragraph{Framework}
We have a set of \emph{news items} $A=\{a_1,...,a_m\}$, each having a specific \emph{cost} $C: A\rightarrow \mathbb{R}$, a set of \emph{recommended items} $W\subseteq A$, a set of \emph{consumers} $N=\{n_1,...,n_n\}$, a \emph{profile of preferences} over the set of items $\mathbf{R}\in \mathcal{L}^n$ and a \emph{budget} $B\in \mathbb{R}_{\geq 0}$.\\
Further, there is a \emph{value function}, akin to a Borda vector, indicating how much a consumer values an option in her ballot amongst the recommended items. It takes as an input the consumer, the profile and an element of $A$,  $V: \mathcal{L}^n \times N \times  A \rightarrow \mathbb{R}$. It is akin to a Borda in the concrete case where $V$ outputs the value $m-1$ for all consumers' candidates in the top position, $m-2$ for the candidates in the second position and so forth. This is very general and is perhaps more conventiently expressed with vector notation, but I could not figure out how to express the maximization problem that way, so I hope the idea is clear.\\

\haukur{I find the value definition a bit too general. It allows for a voter to base her utilities on other people's perferences and I see this as opening a plethora of possibilities and expressibility which we do not need. We could rather assume that consumers' utilities are independent of each other and only based on the item, not considering what other itemes are in $W$. Essentially, each consumer $i$ assigns a value to item $a_j$. For the specific Borda case, $u_i(a_j)=m-1$ if $a_j$ is the item $i$ likes the most.}

What we try to maximize is the sum of all the consumers' values by choosing $W$ (of course this can be recast as a minimization problem by adjusting $V$). For each consumer we only count the value from the news items that are actually in the recommended set:\\
\begin{equation}
\max_W \sum_{j=1}^n \sum_{i=1}^m \one [a_i\in W] V(\mathbf{R}, n_j, a_i)
\end{equation}

Of course, Equation 1 is trivially solved by $W=A$. But the interest in solving it comes from adding the budget constraint.
\begin{equation}
\max_W \sum_{j=1}^n \sum_{i=1}^m \one [a_i\in W] V(\mathbf{R}, n_j, a_i) \text{ subject to } \sum_{a_i \in W} C(a_i) \leq B
\end{equation}
\end{document}