\documentclass{article}
\usepackage[utf8]{inputenc}
\title{Multiwinner rules: proofs of axioms}
\begin{document}
\maketitle

\section{Unanimity axiom}
I will show one counterexample for both utility maximization and $\theta$-minimization rules.

Consider elections with budget $B=10$, a set of items with corresponding costs $X= \{x_1:9, x_2:2, x_3:2, x_4:2, x_5:2\}$ and a following profile of users:
\begin{itemize}
\item 2 voters: $x_1 > x_2 >x_3 > x_4 > x_5$
\end{itemize}
 Notice that following the unanimity axiom $\{x_1 \}$ should be elected. This is the case because it fits into the budget, no other option can be added to this set not exceeding it and all voters rank it on top. But it is not the set chosen neither by the utility maximization nor by $\theta$-minimization rule.
 
\begin{enumerate}
\item The set $\{x_2, x_3, x_4 x_5 \}$ also fits into the budget, but it clearly generates more utility than $\{ x_1\}$ (it gets 12 Borda points while $\{ x_1\}$ only gets 8). So it will rather be elected than $\{x_1 \}$. 
\item Notice that as $x_1$ wins against any other item in a pairwise contest, $\{x_1 \}$ is granted $\theta$ of 100 \%. Therefore, any other set satisfying the budget will have a lower $\theta$. So set $\{x_2, x_3, x_4 x_5 \}$ will rather be elected than $\{x_1 \}$.
\end{enumerate}
\section{Homogeneity axiom}
I will show that homogeneity holds for both utility maximization and $\theta$-minimization rules.
\subsection{Utility Maximization}
Consider any set of voters $V$ and a budget $B$. Consider a ranking of subsets of options satisfying $B$ based on their generated utility:
$$ u(W^*) > u(W_1) > \dots > u(w_n)$$
Now consider an election with a profile of voters $t*V$ and budget $B$. Further notice that for any $W_i$, utility of $W_i$ for profile $t*V$ amounts to $t*u(W_i)$. Hence, the hierarchy of utilities is preserved and therefore $W^*$ remains a winner.
\subsection{$\theta$-minimization rules}
Again, consider any set of voters $V$ and a budget $B$. Also, let $F(V, B) = W$. Notice that in the profile $t*V$ all options received support which does not affect the proportions of their generated utility. Hence, the results of pairwise contests are preserved. So, $W$ remains the best subset satisfying $B$ in terms of $\theta$ minimization. So $F(V, B) = F(t*V, B)$.

\section{Monotonicity}

\subsection{Utility Maximization}

Consider a profile of voters and alternatives $R$ a cost function $C$ and a budget $B$. Let $F$ be the utility maximization rule and let $W\in F(R,B)$. Denote by $u_R$ the utility function that is defined in the obvious way for alternatives as well as sets of alternatives. Consider an alternative $c\in W$ and some $a\in A$ and reader $i\in N$  s.t. $a\succ_i c$ and $a,c$ are immediate neighbours on $i$'s ballot. Denote by $R'$ the profile that coincides with R except that $c\succ_i a$ and $a,c$ are immediate neighbours on $i$'s ballot.

Then we claim that
\begin{enumerate}
\item $c\in W'$
\item $W\in F(R',B)$ if $a\notin W$
\end{enumerate}
Proof:

\begin{enumerate}
\item Note that either $c,a \in W$ (I) or $a\notin W$ (II). For (II) the desired result is immediate. For (I), note that from the way we defined $R'$ by a mere swap it follows that $u_{R'}(\{a,c\})=u_R(\{a,c\})$. Note in addition that if $a\in W'$ then a fortiori $c\in W'$ since $u_{R'}(b)=u_R(b)$ for all $b\in W\setminus \{a,b\}$. Hence suppose that $a,c\notin W'$. Then, since $W'$ is utility maximizing, there must be some set of alternatives $X\subseteq A\setminus W$ s.t. $C(X)\leq C(\{a,c\})$ and $u_{R'}(W\setminus)\{a,c\}\cup X)>u(W)$. Equivalently \[u_{R'}(W)-u_{R'}(\{a,c\})+u_{R'}(X)>u_{R'}(W)\] But $u_{R'}(W)=u_{R}(W)$ since $u_{R'}(\{a,c\})=u_R(\{a,c\})$ and $u_R'(w)=u_R(w)$ for all $w\in W\setminus \{a,c\}$. And $u_R'(X)=u_R(X)$ since nothing outside $W$ was changed. Hence \[u_V(W)-u_V(\{a,c\})+u_V(X)>u_R(W)\] But then $W$ is not utility maximising given $B$ - a contradiction.

\item Suppose $b\notin W$. Since $W$ is utility maximizing given $B$, for any $d\notin W$ and any $S$ s.t. $d\in S$ and $C(S)\leq  B$, either $u_R(S)\leq u_R(W)$ (I) or $C_R(S)>C_R(W)$ (II). For (I), since $u_{R'}(S) \leq  u_{R}(S)+ u_{R'}(a)- u_{R}(a)= u_{R'}(W)$ by utility maximization of $W$ and since the same tie-breaking applies, $W\in F(R',B)$. For (II), since $C$ is unchanged, the desired result is immediate.

\end{enumerate}


\end{document}
