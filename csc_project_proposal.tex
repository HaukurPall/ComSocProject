\documentclass[10pt,a4paper, english]{article}
\usepackage[utf8]{inputenc}
\usepackage[T1]{fontenc}
\usepackage{babel}
\author{
    Haukur J{\'o}nnson\\    \texttt{Haukur's email address}
    \and
    Silvan Hungerb{\"u}hler\\   \texttt{Silvan's email address}
    \and
    Grzegorz Lisowki\\  \texttt{Greg's email address}
    \and
    Max Rapp\\  \texttt{maxgrapp@googlemail.com}
}


\title{%
  COMSOC: Project Proposal \\
  \small Grzegorz Lisowki, Max Rapp and Haukur J{\'o}nnson and Silvan Hungerb{\"u}hler }

\usepackage{mathptmx} % "times new roman"
\usepackage{amssymb}
\usepackage{amsmath, amsthm}
\usepackage{amsfonts}
\usepackage{enumitem}
\usepackage{verbatim}
\usepackage{hyperref}
\usepackage{comment}
%\usepackage[margin=1in]{geometry}
\usepackage{float}
\usepackage{bm}
\usepackage{color}

\usepackage[normalem]{ulem}
\date{}

\begin{document}
\maketitle
<<<<<<< HEAD
Given a set of "voters"/readers $N$ a set of news items $A$ and a corresponding profile $R$ we want to fill a set of relevant alternatives $M$. We do this by iterating a resolute voting rule $F:\mathcal{L}(A)^n \rightarrow A$ $m$ times.\\

Questions:\\
- Which axioms should the relevant set $M$ fulfill and to which axioms on the voting rule $F$ do they correspond?
\section*{Introduction: Goal \& Motivation}

In recent political debate, the increasing fragmentation, partizanship and lack of a common factual base in the media is frequently lamented. We would like take a step towards solving this problem by designing a recommendation mechanism for news articles that unlike conventional approaches considers not only individual preferences but is based on an intersubjective relevance criterion. Namely, given a profile of individual preference orders, repeated application of a social choice function will be used fill a set of essential readings. This set should constitute a "common core" that we suggest should form a subset of any recommendation regardless of indivdual preferences.

=======
\section{Introduction: Goal \& Motivation}
>>>>>>> 927403a2e774b2bd0b13e44d0ad4396afb9d5779
Our project aims at better understanding of collective recommendation mechanisms in media settings. We want to use formal tools provided by Social Choice Theory to analyze benefits and drawbacks of various possible ways to determine a set of essential news items - e.g. newspaper articles - for a group, given each member's individual preferences over the topics - eg. politics, sport, business - instantiated by the items.

News consumers only have so much time and cognitive ressources at their disposal, yet receiving information through news requires both of these. At the same time, it is easy to see the desirability of people - as members in a community or citizens in a state- to be on the same page with respect to topics of high concern to other members. It is, therefore, important to get on understanding of how extract a common core of pieces of information from the dispersed interests and pet topics of individuals while, at the same, time respecting the constraints given by attention span and time.

<<<<<<< HEAD
\section{Getting Started}

We have already started to line out the formal details of the intended mechanism. Namely, given a set of readers $N$, a set of news items $A$ and a corresponding profile $R$ we would like to fill a set of essential readings $M$. We do this by applying a resolute voting rule $F:\mathcal{L}(A)^n \rightarrow A$ $m$ times, deleting the winner from the set of alternatives after each election.\\

In addition we have performed a preliminary literature search. As a next step we will read and assess the relevance of this literature for our project. Especially (Skowron 2014) appears promising. We will compare our approach to those taken in the literature and decide on what changes to implement.

We should then proceed to further discuss and formally define the desiderata the essential readings $M$ should fulfil and find out how they are linked to axioms on the voting rule. Criteria that were considered in preliminary discussion include Condorcet-consistency, manipulability, and minimisation of readers' complaint. This will then allow us to make a choice of the best voting rule for our setting.

\begin{enumerate}
\item Introduction: High-Level motivation. Media context
\item Formal Details. Sequential Voting.
\item Possible avenues: Axiomatic \& Optimization Problem
=======
\section{Formal Model}
%Haukur's comment:
% Given a set of voters/readers $N$, a set of topics $A$, a set of news items $I$ in which each items is associated with some topics, a subset of $A$. The voting rule then selects a topic $a \in A$ and selects a news item from $I$ which corresponds to $a$. We want to fill a set of indispensable $M$.
% The problem I saw with the version below is this: Voters only have preferences over topics but we want the outcome to be an article. We can make hard restrictions to this, f.ex. an article is only based on one topic and we have a article representing each topic in I.
%Grzesiek's comment:
%In this fragment you say that a cost is agent dependent. I thought we were talking about an objective cost hierarchy? If we do not stipulate something like that our model indeed becomes more socially plausible, but on the other hand it would make it harder for us to model as we would probably need to aggregate oopinions about costs as well.
The formal model we will base our anaylisis looks as follows:\\
Given a set of voters/readers $N$ a set of candidates/news items $A$ and a corresponding profile $\mathbf{R}$ we want to fill a set of indispensable $M$. We do this by sequentially applying a resolute voting rule $F:\mathcal{L}(A)^n \rightarrow A$ for $|M|=m$ times. \textcolor{red}{For each step in the sequence we remove the previous step's winner. (Do we need to make this restriction already? there might be a voting rule that funny stuff when we do this)} This process is constrained by an attention budget $b\in \mathbb{R}_{\geq 0}$ for each player. Each item $a\in A$ is associated with a cost $c_a\in \mathbb{R_{\geq 0}}$ and if $M=\{a_1, \dots ,a_m\}$ then $\sum_1^m a_i \leq b$. The cardinality of $M$ thus depends on the items' cost and the composition of $M$.

\section{Avenues of Analysis} 
There are two paths we want to explore in terms of analysis:
\begin{enumerate}
\item As our hope ultimately is to come up with an SCF that provides us with a \textit{good} $M$, given some $\mathbf{R}$ in the context of media, we must qualify what \textit{good} means here. The obvious choice for this task are axiomatic tools from Social Choice Theory. We thus want to determine a set of desiderata a set of indispensable items $M$ should satisfy with respect to a profile $\mathbf{R}$. By way of illustration, one such axiom for $M$ could demand that any item that is present in the $90^th$ percentile for all readers, must be present in $M$.
%Grzesiek's comment
%Do we want to be a bit more specific about which axioms we think should be included? 
We also hope to find some interaction between axiomatic choices at the level of SCF and $M$.
\item Instead of taking the axiomatic approach we could view the problem as minimizing some sort of discrepancy between $M$ and what the readers' preferences are, while respecting the budget constraint. Intuitively, such a discrepancy would be captured by a metric that, to give an example, indicates high discrepancy if the unanimous top-priority in the profile is not present in $M$ and lower discrepancy if the unanimous bottom-priority is absent. Our problem then becomes the challenge of finding a voting rule that yields the subset of $A$ - $M$ that minimizes discrepancy while staying within the budget.
\end{enumerate}
\item Furthermore, we can compare the results of k-sequential voting and non-resolute voting rules which give K winners. Similarly as above we can compare the set $M$ given by each procedure using some discrepancy measure in order to evaluate the procedures. Additionally, we can compare the complexity of these two different procedures.
>>>>>>> 927403a2e774b2bd0b13e44d0ad4396afb9d5779
\end{enumerate}

\section{References}
Phd Thesis: https://www.mimuw.edu.pl/~ps219737/phdThesis.pdf
<<<<<<< HEAD

\end{document}
