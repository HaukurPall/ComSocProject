\documentclass[10pt,a4paper, english]{article}
\usepackage[utf8]{inputenc}
\usepackage[T1]{fontenc}
\usepackage{babel}
\author{
    Haukur J{\'o}nnson\\    \texttt{haukurpalljonsson@gmail.com}
    \and
    Silvan Hungerb{\"u}hler\\   \texttt{silvan.hungerbuehler@bluewin.ch}
    \and
    Grzegorz Lisowski\\  \texttt{grzegorz.adam.lisowski@gmail.com}
    \and
    Max Rapp\\  \texttt{maxgrapp@gmail.com}
}


\title{%
  COMSOC: Project Proposal}

\usepackage{mathptmx} % "times new roman"
\usepackage{amssymb}
\usepackage{amsmath, amsthm}
\usepackage{amsfonts}
\usepackage{enumitem}
\usepackage{verbatim}
\usepackage{hyperref}
\usepackage{comment}
%\usepackage[margin=1in]{geometry}
\usepackage{float}
\usepackage{bm}
\usepackage{color}

\usepackage[normalem]{ulem}
\date{}

\newcommand{\haukur}[1]{\textcolor[rgb]{.8,.33,.0}{[TB: #1]}}% prints in orange
\newcommand{\Max}[2]{\textcolor[rgb]{.15,.73,.12}{[Max: #1]}}%green

\begin{document}
\maketitle

\section{Introduction: Goal \& Motivation}

In recent political debate, the increasing fragmentation, partizanship and lack of a common factual base in the media is frequently lamented. We would like take a step towards solving this problem by designing a recommendation mechanism for news articles that unlike conventional approaches considers not only individual preferences but is based on an intersubjective relevance criterion. Namely, given a profile of individual preference orders, repeated application of a social choice function will be used fill a set of essential readings. This set should constitute a "common core" that we suggest should form a subset of any recommendation.

Our project aims at better understanding of collective recommendation mechanisms in media settings. We want to use formal tools provided by Social Choice Theory to analyze benefits and drawbacks of various possible ways to determine a set of essential news items - e.g. newspaper articles - for a group, given each member's individual preferences over the topics - eg. politics, sport, business - instantiated by the items.
\section{Getting Started}

We have already started to line out the formal details of the intended mechanism. Details are provided in Section 3.

In addition we have performed a preliminary literature search. As a next step we will read and assess the relevance of this literature for our project. Especially (Skowron 2014) appears promising. We will compare our approach to those taken in the literature and decide on what changes to implement.

We should then proceed to further discuss and formally define the desiderata the set of essential items should fulfil and find out how they are linked to axioms on the voting rule. This will then allow us to make a choice of the best voting procedure for our setting.

\section{Formal Model}
Preliminarily, we will base our analysis on the following formal model:\\
Given a set of voters/readers $N=\{1,2, ..., n\}$, a set of binary issues $\mathcal{X} = \{X_1 , ... , X_p\}, with $D_i = \{0_i , 1_i\} for each $i$. These binary issues correspond to a set of items/articles $C =\{c_1 , ... , c_p\}, where $X_i = 1_i$ means that item $c_i$ is in the selection $S$. The set of alternatives is then $A = \{0_1, 1_1 \} \times ... \times {0_p , 1_p }. We want the selection $S$ to contain $k$ elements which satisfy $\sum_{x \in S}cost(x) \leq c$. We call a set which satisfies the cost restraints a \textit{possible candidate set}. From that set we apply a voting rule $F:\mathcal{L}(A)^n \rightarrow A$.

\section{Avenues of Analysis}
Once we have specified a basic formal model as a sort of baseline, there are two paths we want to explore to the end of further analysis:
\begin{enumerate}
\item As we ultimately want to come up with an SCF that provides us with a \textit{good} $M$, given some $\mathbf{R}$ in the context of media, we must qualify what \textit{good} means here. The obvious choice for this task are axiomatic tools from Social Choice Theory. We thus want to determine a set of desiderata a set of indispensable items $M$ should satisfy with respect to a profile $\mathbf{R}$. By way of illustration, one such axiom for $M$ could demand that any item that is present in the $90^th$ percentile for all readers, must be present in $M$.

We also hope to find some interaction between axiomatic choices at the level of SCF - as Neutrality, the Pareto or Condorcet principle, for example - and axioms at the level of $M$. Especially the latter axioms yet need to be identified and described formally for such a study to take place.
\item News consumers only have so much time and cognitive ressources at their disposal, yet receiving information through news requires both of these. At the same time, it is easy to see the desirability of people - as members in a community or citizens in a state- to be on the same page with respect to topics of high concern to other members. It is, therefore, important to get on understanding of how extract a common core of pieces of information from the dispersed interests and pet topics of individuals while, at the same, time respecting the constraints given by attention span and time.

Formally speaking, the process of filling $M$ could be constrained by an attention budget $b\in \mathbb{R}_{\geq 0}$ for all players. Each item $a\in A$ is associated with a cost $c_a\in \mathbb{R_{\geq 0}}$ and if $M=\{a_1, \dots ,a_m\}$ then $\sum_1^m a_i \leq b$. The cardinality of $M$ thus depends on the items' cost and the composition of $M$.

Instead of taking the axiomatic approach we can then view the problem as minimizing some sort of discrepancy between $M$ and what the readers' preferences are, while respecting the budget constraint. Intuitively, such a discrepancy would be captured by a metric that, to give an example, indicates high discrepancy if the unanimous top-priority in the profile is not present in $M$ and lower discrepancy if the unanimous bottom-priority is absent. Our problem then becomes the challenge of finding a voting rule that yields the subset of $A$ - $M$ that minimizes discrepancy while staying within the budget.

\section{Issues to Address}
\begin{enumerate}
\item Are there dependencies between the elements of A? \haukur{I think it would be nice to be able to make that assumption, but it will make the model more complex and the benefits to our model of more expressibility are not that great (in my view). What do you guys think?} \Max{I think this is something one would definitely want in an application ready algorithm, however with dependencies on topics (like Silvan suggested) not single articles. However, for the baseline I think we should not overcomplicate things. It would be a good thing to add later. }
\item Piotr's thesis. \haukur{We should probably only stick to: Finding a Collective Set of Items: From Proportional Multirepresentation to Group Recommendation}
\item Multiwinner voting rules. \haukur{We need to find papers about this}\Max{see my additions to the bibliography. The handbook section 9.3.2 is a good starting point for further sources}
\item First meeting. We need to identify a small set of relevant papers and eplain how they relate to our work. As well as clarify our model better. \haukur{Like above I think we need to find some papers on multiwinner voting rules as well as get a clearer understanding of Piotr's work.}
\end{enumerate}

\end{enumerate}

\end{document}
