\documentclass[10pt,a4paper, english]{article}
\usepackage[utf8]{inputenc}
\usepackage[T1]{fontenc}
\usepackage{babel}
\author{
    Haukur J{\'o}nnson\\    \texttt{haukurpalljonsson@gmail.com}
    \and
    Silvan Hungerb{\"u}hler\\   \texttt{silvan.hungerbuehler@bluewin.ch}
    \and
    Grzegorz Lisowski\\  \texttt{grzegorz.adam.lisowski@gmail.com}
    \and
    Max Rapp\\  \texttt{maxgrapp@gmail.com}
}


\title{%
  COMSOC: Meeting agenda}

\usepackage{mathptmx} % "times new roman"
\usepackage{amssymb}
\usepackage{amsmath, amsthm}
\usepackage{amsfonts}
\usepackage{enumitem}
\usepackage{verbatim}
\usepackage{hyperref}
\usepackage{comment}
%\usepackage[margin=1in]{geometry}
\usepackage{float}
\usepackage{bm}
\usepackage{color}

\usepackage[normalem]{ulem}
\date{}

\newcommand{\haukur}[1]{\textcolor[rgb]{.8,.33,.0}{[TB: #1]}}% prints in orange
\newcommand{\Max}[2]{\textcolor[rgb]{.15,.73,.12}{[Max: #1]}}%green

\begin{document}
\maketitle

\section{What have we done so far (15 minutes)}
\begin{itemize}
\item We have discussed several voting rules and desirable properties of the winner set and how they relate.

Need to explain our model. Voters/consumers, candidates/articles, preference orders, costs function.

\item We have done a survey of the literature, especially looking at multiwinner rules, condorcet based methods and budgeted social choice.

Explain the literature we read on k-winner voting rules and explain that k-winners are not our criteria, but rather budget.

Why not use C\&C or Monroe. We considered Skowron but decided that representation is also not important since we assume that everyone reads everything.

Elkind 2017 our primary paper now

\item We have prepared an outline to our project.
\end{itemize}

\section{What issues can you help us with? (25 minutes)}
\begin{itemize}
\item We would like feedback on voting rules. In particular, we abandoned multiwinner rules. Is this the correct way to go?
\item The Schultze rule is interesting, but we haven't discussed it in class. Why not?
\item Our voting rules.
\item Properties of the winning set, including our proposal. What conditions should such a set meet?

Talk about interesting axioms which Greg found

\item Should we focus on simulations or proofs? Should we generate data or attempt to find it?

\end{itemize}

\section{Plans for completing the project? (15 minutes)}
\begin{itemize}
\item Implement the voting rules.
\item Gather or generate some data.
\item Compare performance of voting rules w.r.t. desirable properties.
\item Run simulations and, if appliccable, attempt proofs.
\end{itemize}

Ulle's comments.

Axiom idea:
10\% of ppl really like some topic and want to force it in, if it is less than 10\% of the budget it should be in

The difference between properties of outcomes (can be simulated) and axioms.

We need to consider cases in which we have 3 (expensive) stories or 6 (cheap) stories, perhaps consider utility per minute of each article.

How to evaluate our framework as a service, do we want many people to be happy? Do we want most people to be fairly happy?

Mention multiwinner rules, axioms in related work.

Produce graphs, try to not have too many parameters and try to read some qualitative information from the graphs. Cost distribution.

Using slack variables (empty minutes have some value).

Handfull of good things, rather than lots of everything.

Conclusion: What we learnt, under these conditions this works well etc.
\end{document}
