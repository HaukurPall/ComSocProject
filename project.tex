% These are the instructions for authors for IJCAI-17.
% They are the same as the ones for IJCAI-11 with superficical wording
%   changes only.

\documentclass{article}
% The file ijcai17.sty is the style file for IJCAI-17 (same as ijcai07.sty).
\usepackage{ijcai17}

% Use the postscript times font!
\usepackage{times}

% the following package is optional:
%\usepackage{latexsym}

% Following comment is from ijcai97-submit.tex:
% The preparation of these files was supported by Schlumberger Palo Alto
% Research, AT\&T Bell Laboratories, and Morgan Kaufmann Publishers.
% Shirley Jowell, of Morgan Kaufmann Publishers, and Peter F.
% Patel-Schneider, of AT\&T Bell Laboratories collaborated on their
% preparation.

% These instructions can be modified and used in other conferences as long
% as credit to the authors and supporting agencies is retained, this notice
% is not changed, and further modification or reuse is not restricted.
% Neither Shirley Jowell nor Peter F. Patel-Schneider can be listed as
% contacts for providing assistance without their prior permission.

% To use for other conferences, change references to files and the
% conference appropriate and use other authors, contacts, publishers, and
% organizations.
% Also change the deadline and address for returning papers and the length and
% page charge instructions.
% Put where the files are available in the appropriate places.

\title{Collective Recommendations}
\author{
    Haukur J{\'o}nsson \\    \texttt{haukurpalljonsson@gmail.com}
    \and
    Silvan Hungerb{\"u}hler \\   \texttt{silvan.hungerbuehler@bluewin.ch}
    \and
    Grzegorz Lisowski \\  \texttt{grzegorz.adam.lisowski@gmail.com}
    \and
    Max Rapp \\  \texttt{maxgrapp@gmail.com}
}

\begin{document}

\maketitle

\begin{abstract}
Some abstract
\end{abstract}

\section{Section}\label{intro}

This is how you {\it do italics}.

And this is how you {\em emphasize}

\subsection{Subsection}

Reference to Section~\ref{intro}

{\tt code text}

\begin{quote}
And a quote
\end{quote}

Footnote\footnote{blurg}

A bold {\bf letter}

\subsubsection{References}

Referencing ~\shortcite{Elkind2015} in text. Speaking about work ~\cite{Elkind2015}.

\subsection{Haukur's work}

One idea on scoring vectors to, perhaps, capture more realistic preferences.

Three voting rules.

\subsubsection{Capturing interest groups}

Instead of considering the borda scoring vector when maximizing utility we might want to use a vector which gives the
first candidate marginally more points than the second candidate and so on until we reach the middle candidate then the margin
 starts to grow again until it reaches what it was in the beginning. This allows voters to select a few items which they
 "really like" and a few items which they "really hate". Think of $-x^3$.

\subsubsection{Schulze rule}

Has many nice properties and is closely related to the $20\%$ minimal requirement.

Computationally feasible.

\subsubsection{$\Theta$-Smith set}

The $\Theta$-Smith set is the smallest non-empty set of candidates s.t. each member of the set defeats every other member outside
 the set in $\Theta \%$ of cases.

 The $\Theta$-Smith set is a Condorcet extension.

A suggestion of an algorithm which selects a "good" $\Theta$-Smith set. Compute the Smith set for $\Theta=50$
(the normal Smith set), check if it satisfies the cost restricitons. Second, check the upper-bound by finding the Smith
set of $\Theta=100$, if it satisfies the cost restriction, select it. Depending on the case the cost restriction was not
satsfied we need to iterate over the interval in which the cost restriction break, start iterating from $\Theta=50$ and
 increase/decrease $\Theta$ by $(100-50)/2=25$ or $(50-0)/2=25$ until a set is found which satisfies the cost
 restrictions (when found, there might be multiple.).

Computationally feasible.

\subsubsection{K-plurality rule}

The {\em k-plurality rule} with $k<|A|$ is a {\em positional scoring rule} with the same scoring vector as the normal
{\em plurality rule}, $(1,0, \dots, 0)$, but instead of electing the alternative(s) with the highest score it elects the
alternative(s) with the highest score, if the number of winners is strictly less than $k$ then the alternative(s) with
the second highest score is elected. This is done until $|W| \geq k$. If $|W| > k$ then a tie-breaker should be applied
on the last iteration.

Computationally feasible.

%% The file named.bst is a bibliography style file for BibTeX 0.99c
\bibliographystyle{named}
\bibliography{references}

\end{document}

